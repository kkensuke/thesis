% figure caption
\usepackage{caption}
\captionsetup[figure]{labelformat=empty}
% font family
\usepackage{helvet}
% use gothic font for japanese
\renewcommand{\kanjifamilydefault}{\gtdefault}


% theme
\usetheme{Madrid}
\usefonttheme{professionalfonts}
\useoutertheme[height=0cm,width=1.5cm,left]{sidebar}


% frame number
\setbeamertemplate{frametitle}{
    \nointerlineskip
    \begin{beamercolorbox}[wd=\paperwidth,ht=2.25ex,dp=0.75ex]{frametitle} % set ht
        \hspace*{1ex}\insertframetitle
        % \hfill\insertframenumber/\inserttotalframenumber\hspace*{8ex}
        \hfill\insertframenumber/{27}\hspace*{8ex}
    \end{beamercolorbox}
}


% hides nvigation buttons at bottom
\setbeamertemplate{navigation symbols}{}


% Remove title and name from sidebar
\makeatletter
\setbeamertemplate{sidebar \beamer@sidebarside}%{sidebar theme}
{
    \beamer@tempdim=\beamer@sidebarwidth%
    \advance\beamer@tempdim by 10pt%
    \insertverticalnavigation{\beamer@sidebarwidth}%
    \vfill
    \ifx\beamer@sidebarside\beamer@lefttext%
    \else%
    \usebeamercolor{normal text}%
    \llap{\usebeamertemplate***{navigation symbols}\hskip0.1cm}%
    \vskip5pt%
    \fi%
}%
\makeatother


% show toc at the beginning of each section
\AtBeginSection[]
{
    \begin{frame}
        \frametitle{Contents}
        \tableofcontents[currentsection]
    \end{frame}
}


% display transparently
\setbeamercovered{transparent}


% table style
% This sets the thickness of the borders of the table.
\setlength{\arrayrulewidth}{0.5mm}
% The space between the text and the left/right border of its containing cell
\setlength{\tabcolsep}{18pt}
% The height of each row is set to 1.5 relative to its default height.
\renewcommand{\arraystretch}{2.5}


% define colors
% ---------------------------------------------------------------------------- %
\definecolor{red}{rgb}{0.9,0.30,0.30}
\definecolor{blue}{rgb}{0.32,0.66,0.82}
\definecolor{darkblue}{rgb}{0.2,0.4,0.6}
\definecolor{green}{rgb}{0.47,0.72,0.29}
\definecolor{darkgreen}{rgb}{0.25,0.42,0.21}
\definecolor{yellow}{rgb}{0.95,0.85,0.25}
\definecolor{darkyellow}{rgb}{0.75,0.65,0.05}
\definecolor{orange}{rgb}{0.95,0.55,0.25}
\definecolor{darkorange}{rgb}{0.75,0.35,0.05}
\definecolor{purple}{rgb}{0.75,0.55,0.85}
\definecolor{darkpurple}{rgb}{0.55,0.35,0.65}
\definecolor{brown}{rgb}{0.75,0.55,0.25}
\definecolor{darkbrown}{rgb}{0.55,0.35,0.05}
\definecolor{pink}{rgb}{0.95,0.55,0.75}
\definecolor{darkpink}{rgb}{0.75,0.35,0.55}
\definecolor{grey}{rgb}{0.55,0.55,0.55}
\definecolor{darkgrey}{rgb}{0.35,0.35,0.35}
% ---------------------------------------------------------------------------- %


% set colors
% ---------------------------------------------------------------------------- %
\setbeamercolor{structure}{fg=blue}
\setbeamertemplate{blocks}[rounded][shadow=false]
\setbeamercolor{block title alerted}{bg=red, fg=white}
\setbeamercolor{block title example}{bg=green, fg=white}


% define blocks
% ---------------------------------------------------------------------------- %
\addtobeamertemplate{proof begin}{
    \setbeamercolor{block title}{bg=grey, fg=white}
}{}

\newenvironment<>{note}[1]{
    \setbeamercolor{block title}{bg=blue, fg=white}
    \begin{block}{Note}#1}{\end{block}}
\newenvironment<>{warning}[1]{
    \setbeamercolor{block title}{bg=red}
    \begin{block}{Warning}#1}{\end{block}}
\newenvironment<>{important}[1]{
    \setbeamercolor{block title}{bg=orange}
    \begin{block}{Important}#1}{\end{block}}

\newenvironment<>{theorem}[1]{
    \setbeamercolor{block title}{bg=darkblue, fg=white}
    \begin{block}{Theorem}#1}{\end{block}}
% \newenvironment<>{proof}[1]{
%     \setbeamercolor{block title}{bg=grey, fg=white}
%     \begin{block}{証明}#1}{\end{block}}
\newenvironment<>{definition}[1]{
    \setbeamercolor{block title}{bg=grey}
    \begin{block}{Definition}#1}{\end{block}}
\newenvironment<>{proposition}[1]{
    \setbeamercolor{block title}{bg=darkblue}
    \begin{block}{Proposition}#1}{\end{block}}
\newenvironment<>{lemma}[1]{
    \setbeamercolor{block title}{bg=darkblue}
    \begin{block}{Lemma}#1}{\end{block}}
\newenvironment<>{corollary}[1]{
    \setbeamercolor{block title}{bg=darkblue}
    \begin{block}{Corollary}#1}{\end{block}}
\newenvironment<>{remark}[1]{
    \setbeamercolor{block title}{bg=blue}
    \begin{block}{Remark}#1}{\end{block}}
% ---------------------------------------------------------------------------- %