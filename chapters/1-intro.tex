\chapter{序論}

% 背景・不足・目的・結果・構成

\section{研究背景}
量子コンピューターの研究は、1980年代に Paul Anthony Benioff、David Deutsch、Richard Feynman などらが量子力学に基づいたコンピューターによる計算を提案したところから始まった\cite{benioff1980computer,deutsch1985quantum,feynman1982simulating}。
量子コンピューターは、量子力学の性質を帯びる系(2準位系)を情報の最小単位として用いるコンピューターである。その最小単位を量子ビットと呼ぶ。量子ビットは、重ね合わせやエンタングルメントといった量子力学的性質を持つ。これらの性質を利用することで、量子コンピューターは従来のコンピューターよりもある種の計算を効率的に行うことができるのではないかと期待されている。以降では、従来のコンピューターのことを古典コンピューターと呼ぶ。

1990年代に入るとさまざまな量子アルゴリズムが提案されるようになり、とりわけ、Peter Shor が、素因数分解問題を多項式時間で解く量子アルゴリズム\cite{shor1997polynomial-time}(Shor のアルゴリズム)を発表したことで、量子コンピューターの研究が注目されるようになった。というのも、古典コンピューターでは素因数分解問題を解くための計算量は指数関数的に増大するが、その計算量的な難しさこそが、RSA 暗号の安全性を保証しているからである。

それ以来、より一層研究が進められ、量子計算理論の構築・量子アルゴリズム開発・量子コンピューターのソフトウェア/ハードウェア開発など、多方面の知識と技術が向上していった。そして、2010年代には D-wave の量子アニーリングマシン、Google と IBM の超伝導量子コンピューターが登場した。最近では、Google や IBM が量子超越に関する論文~\cite{arute2019quantum,kim2023evidence}を発表したことで、より量子コンピューターの研究が注目されている。

Shor のアルゴリズム以外にも、非構造的なデータベースから欲しいデータを見つける量子アルゴリズム\cite{grover1996fast}(1995年:Grover のアルゴリズム)や、線型方程式を解く量子アルゴリズム\cite{harrow2009quantum}(2009年:HHL アルゴリズム)などの効率的な量子アルゴリズムが提案された。
これらのアルゴリズムは、量子並列性・エンタングルメントといった量子力学的性質をうまく利用することで、古典計算より効率的であると期待されている。

これらのアルゴリズムを実装するには、大規模で誤り訂正可能な量子コンピューターが必要である。しかしながら、現在開発されている量子コンピューターは、中規模(数百量子ビット)であり、ノイズが大きく、量子ビット同士の相互作用は近接のものだけである。そのような量子コンピューターを \ruby{NISQ}{ニスク} (Noisy Intermediate-Scale Quantum Computer)と呼ぶ\cite{preskill2018quantum}。NISQは誤り訂正機能を持たず、ノイズの影響を受けやすいため、使用する量子ゲートの数をなるべく少なくする必要がある。

そのような量子ビット数と量子ゲート数の制約の中でも動作する量子アルゴリズムとして、変分量子アルゴリズム(Variational Quantum Algorithm, VQA)\cite{cerezo2021variational}が注目されている。なぜなら、このアルゴリズムはノイズに強く、少ない量子ゲートで実装できるという特徴があるからである。
変分量子アルゴリズムは、古典コンピューターと組み合わせて最適化問題を解くアルゴリズムである。最適化問題とは、ある関数の値を最小化(あるいは最大化)するようなパラメーターの値を求める問題である。
変分量子アルゴリズムにおいては、パラメーター化された量子状態を試行関数として用いる。そして、その試行関数のパラメーターを古典コンピューター上で更新することで、コスト関数と呼ばれる関数の値を最小化するように学習する。
初めは、量子化学計算における変分量子固有値ソルバー(VQE)~\cite{peruzzo2014variational}や、組み合わせ最適化問題における変分量子近似最適化(QAOA)~\cite{farhi2014quantum}のアルゴリズムとして提案された。
しかし、変分量子アルゴリズムは、問題をコスト関数に書き換えることができれば、どのような問題にも適用できる。そのため、応用範囲は極めて広く、量子化学計算\cite{peruzzo2014variational,mcardle2019variational,yuan2019theory}、量子シミュレーション\cite{georgescu2014quantum,banuls2020simulating}、量子コンパイリング\cite{khatri2019quantum-assisted,sharma2020noise}、機械学習\cite{mitarai2018quantum,farhi2018classification,havlicek2019supervised,schuld2019quantum}といった様々な分野で研究されている。このように、変分量子アルゴリズムは NISQ 時代における有望なアルゴリズムとして注目されている。

そんな中、変分量子アルゴリズムにおいて、バレンプラトーと呼ばれるコスト関数の勾配消失が起こりうることが判明した~\cite{mcclean2018barren}。バレンプラトーとは、コスト関数の勾配の平均が0で、勾配の分散が量子ビット数に関して指数関数的に小さくなる現象のことである。よって、バレンプラトーが起きると、コスト関数の勾配を正確に推定するために指数関数的に多くの測定が必要となり、効率的な学習ができなくなる~\cite{arrasmith2021effect}。バレンプラトーを引き起こす原因は1つではなく、量子変分回路の構造~\cite{mcclean2018barren,marrero2021entanglement,holmes2022connecting}や、測定するオブザーバブル~\cite{cerezo2021cost}、ノイズ~\cite{wang2021noiseinduced}、訓練データの入力~\cite{thanasilp2021subtleties,thanasilp2022exponential}など複数の原因が先行研究で指摘されている。

\section{本研究の目的と結果}
近年、機械学習はさまざまな分野で著しい成果を上げているが、その一方で、高い計算コストが問題となっている。例えば、2022年に公開された ChatGPT-3 は最大1750億個のパラメーターを持つ大規模言語モデルである。そのような巨大なモデルを学習するためには、多くの計算機と数週間から数ヶ月に及ぶ学習が必要とされる。
このように、機械学習の計算コストは膨大であり、その計算コストを削減することは重要な課題である。そこで、機械学習の計算を量子コンピューター上で行い、計算を効率化することが期待されている。この研究分野を量子機械学習\cite{biamonte2017quantum}と呼ぶ。

量子計算の機械学習への応用は、変分量子アルゴリズムの登場以前からあり、HHL アルゴリズムを応用したアルゴリズムが提案されていた。しかし、NISQ 上での実装が難しいため、最近では変分量子アルゴリズムを応用した機械学習アルゴリズムが次々と提案されるようになった。量子ニューラルネットワーク(QNN)~\cite{farhi2018classification,mitarai2018quantum}、量子畳み込みニューラルネットワーク(QCNN)~\cite{cong2019quantum}、量子カーネル法\cite{havlicek2019supervised,schuld2019quantum}などがその例である。

本研究は、変分量子アルゴリズムの文脈における教師あり量子機械学習の効率化に焦点を当てた。
量子機械学習アルゴリズムの効率化においては、訓練データとその入力方法(以降、二つを合わせて「データ入力」と呼ぶ)が重要であることが指摘されており\cite{huang2021power,caro2021encodingdependent}、先ほど述べたバレンプラトーとの関連も見出された\cite{thanasilp2021subtleties,thanasilp2022exponential}。しかしながら、多くのバレンプラトー研究は主に学習回路の構造に着目してきたため、データ入力がバレンプラトーに与える影響は十分には理解されていない。
そこで、特に、量子機械学習に特有のデータ入力がバレンプラトーにどのような影響を与えるのか評価した。バレンプラトーは複数の原因によって生じるため、データ入力によるバレンプラトーの評価においては、他の原因を排除した量子回路の設定の下で解析を行った。
その結果、データ入力後の量子状態のエンタングルメントが多い、あるいは、訓練データを入力するための量子回路(以降、入力回路と呼ぶ)の表現能力が高いほど、コスト関数の勾配の分散の上界が小さくなり、バレンプラトーにつながることが明らかになった。そして、具体的な入力回路を用いたシミュレーションの結果と解析計算の結果を比較することで、解析計算の妥当性を確認した。
また、量子回路に入力する訓練データにガウス分布を仮定し、絶対誤差のコスト関数の勾配の分散の下界についても解析を行った。その結果、絶対誤差のコスト関数の勾配の分散の下界において、訓練データの分散が大きな役割を果たすことが分かった。

% [4] “IBM Makes Quantum Computing Available on IBM Cloud to Accelerate Innovation,” (2016), press release at https://www-03.ibm.com/press/us/en/pressrelease/49661.wss.

% IBMQ, (2019), “Qiskit: An open-source framework for quantum computing,” 10.5281/zenodo.2562110.


\section{本論文の構成}
本論文の構成は以下の通りである。
\begin{itemize}
    \item 第\ref{chap:quantum-circuit}章では、量子回路の要素と量子計算の基礎について述べる。
    \item 第\ref{chap:vqa}章では、変分量子アルゴリズムとバレンプラトーについて述べる。
    \item 第\ref{chap:haar}章では、バレンプラトーの解析に必要となるハール分布とユニタリ $t$--デザインについて述べる。
    \item 第\ref{chap:upper-bound}章では、量子機械学習におけるバレンプラトーの解析として、コスト関数の勾配の分散の上界をデータ入力の観点から導出する。
    \item 第\ref{chap:lower-bound}章では、量子機械学習におけるバレンプラトーの解析として、コスト関数の勾配の分散の下界をデータ入力の観点から導出する。
    \item 付録には、本論文で用いる定理・補題の証明、その他の補足を記す。
\end{itemize}
