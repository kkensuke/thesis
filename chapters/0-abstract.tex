\thispagestyle{empty}
\null\vspace{\fill}

\begin{center}
    \textbf{\Large 概要}
\end{center}
 現在、機械学習はあらゆる分野で応用されている。しかし、機械学習の計算コストは膨大であり、その計算コストを削減することは重要な課題である。そこで、機械学習の計算を量子コンピューター上で行い、計算を効率化することが期待されている。この研究分野を量子機械学習と呼ぶ。

現在開発されている量子コンピューターは、数百量子ビット程度で、ノイズが大きく、量子ビット同士の相互作用が近接のものだけである。そのような量子コンピューターを Noisy Intermediate-Scale Quantum Computer (\ruby{NISQ}{ニスク})と呼ぶ。NISQはノイズの影響を受けるため、使用できる量子ゲート数が制限される。
そのような制約の中でも動作する量子アルゴリズムとして、変分量子アルゴリズムが注目されている。
変分量子アルゴリズムは、古典コンピューターと組み合わせて最適化問題を解くアルゴリズムである。
特に、変分量子アルゴリズムの枠組みで、量子機械学習の効率化も研究されている。
しかし、変分量子アルゴリズムにおいては、コスト関数の勾配消失問題が最適化における大きなボトルネックとなっている。

本研究は、量子機械学習における識別モデルのコスト関数の勾配消失問題について、訓練データの入力が与える影響を解析した。その結果、訓練データ入力後の量子状態のエンタングルメントの大きさや訓練データの入力を行うための量子回路の表現能力の高さが最適化を困難にしうることが分かった。そして、その解析結果が数値計算と矛盾しないことを確認した。
また、訓練データがガウス分布に従う場合について、その訓練データの分散がコスト関数の勾配のスケーリングにおいて重要な役割を果たすことを示した。

\vspace{\fill}
